\documentclass[10pt,a4paper]{article}
\usepackage[utf8]{inputenc}
\usepackage{amsmath}
\usepackage{amsfonts}
\usepackage{amssymb}
\usepackage{comment}
\usepackage{commath}
\begin{document}
\title{Pattern generator \\coupling analytic footprint MPC formulation and whole body control}
\author{Thomas Flayols}
\maketitle
\section*{MPC part}

    Follow a COM velocity (evaluated at each phase switching)
    \begin{equation}\label{min_velocity}
        \begin{pmatrix} 
             F_u(t)        		& 0 	 		& 0 &0& \hdots & 0 \\
             {F_x}^1 F_u(t) 		& F_u 	 		& 0 &0& \hdots & 0 \\
             {F_x}^2 F_u(t) 		& {F_x}^1 F_u 	& F_u &0& \hdots & 0 \\ \\
             \vdots	 			&\vdots 	& \vdots & \vdots &  \ddots & 0 \\ \\
             {F_x}^{N_s-1} F_u(t) 	&  {F_x}^{N_s-2}	& \hdots& {F_x}^2 F_u& {F_x}^1 F_u  & F_u \\
        \end{pmatrix}\begin{pmatrix} 
            p_0 \\
            \vdots \\
            p_{N_s}
        \end{pmatrix} - \begin{pmatrix} 
            \dot{c}^*-F_x(t) c_0\\
            \dot{c}^*-F_x(t){F_x}^1 c_0\\
            \vdots \\
            \dot{c}^*-F_x(t){F_x}^{N_s-1} c_0\\
        \end{pmatrix}
    \end{equation}

    With a reasonable step size
    (to be replaced with inegality constrains)
    \begin{equation}\label{min_step_dist}
        \begin{pmatrix} 
             \beta    & -\beta 	 	& 0 &0& \hdots & 0 \\
             0&\beta    & -\beta 	 	& 0 & \hdots & 0 \\
             \vdots	 			&\vdots 	& \vdots & \vdots &  \ddots & 0 \\ 
             0       		&  	 		&    \hdots&  & \beta    & -\beta \\
        \end{pmatrix}
        \begin{pmatrix} 
            p_0 \\
            \vdots \\
            p_{N_s}
        \end{pmatrix}- \begin{pmatrix} 
            \beta \Delta\\
            \beta (-1)\Delta\\
            \vdots \\
            \beta(-1)^{N_s-1}\Delta\\
        \end{pmatrix}
    \end{equation}




    ZMP should be closed to the center of support foot (p0-p0*).

    Landing foot position should not change to much when the flyingfoot is about to land (p1-p1*).
    \begin{equation}\label{p0_p1}
        \begin{pmatrix} 
             \Gamma    & 0 	 	& 0 &0& \hdots & 0 \\
             0&\Gamma_2    & 0	 	& 0 & \hdots & 0 \\
            \end{pmatrix}
        \begin{pmatrix} 
            p_0 \\
            \vdots \\
            p_{N_s}
        \end{pmatrix}- \begin{pmatrix} 
            \Gamma     {p_{0}}^*\\
             \Gamma_2  {p_{1}}^*\\
        \end{pmatrix}
    \end{equation}

Note: the step is divided in two phases, one called adaptative where there is no constraints on landing area ($\Gamma_2=0$). Durring this phase, the foot trajectory adapts to the the landing goal position.
In the second phase (called non-adaptative) the foot landing area must converge closed to the last landing area preicted in the first phase. $\Gamma_2=0$ goes from zero to $\Gamma$ to have the same effect that the next constrains on zmp.
\section*{FullBody part}
    Altiutude of com is regulated whith a PD, so as the support foot, the torso and the altitude of the flyingfoot.
    A posture task is given, with a low gain so it behaves like a lower priority task.
    \begin{equation}\label{FullBody}
        \begin{pmatrix} 
             [\bold{J}_{com}]_z\\
             [\bold{J}_{torso}]_{rx,ry,rz}\\
             [\bold{J}_{flyingFoot}]_{z,r_x,r_y,r_z}\\
              \bold{J}_{supportFoot} \\
              {\delta}\bold{J}_{Posture} \\
        \end{pmatrix}
        \bold{\ddot{q}}-
            \begin{pmatrix} 
             [-{K_p} \varepsilon_{com}        - {K_d}\dot{\varepsilon}_{com}        - \bold{\dot{J}}_{com} \dot{q}]_z\\
             [-{K_p} \varepsilon_{torso}      - {K_d}\dot{\varepsilon}_{torso}      - \bold{\dot{J}}_{torso} \dot{q}]_{r_x,r_y,r_z} \\
             [-{K_p} \varepsilon_{flyingFoot} - {K_d}\dot{\varepsilon}_{flyingFoot}  -\bold{\dot{J}}_{flyingFoot} \dot{q}]_{z,r_x,r_y,r_z}\\
             -{K_p} \varepsilon_{supportFoot} - {K_d}\dot{\varepsilon}_{supportFoot} -\bold{\dot{J}}_{supportFoot} \dot{q}\\
             {\delta}[-{K_p} \varepsilon_{Posture} - {K_d}\dot{\varepsilon}_{Posture}] \\
            \end{pmatrix}
    \end{equation}

\section*{Coupling contrains}
    The center of mass acceleration in x and y is a linear function of p0 (aka zmp) thanks to the LIP model
    \begin{equation}
        [{J}_{com} \ddot{q}+\dot{J}_{com} \dot{q}]_{x,y}=[a_{com}p_0+b_{com}]_{x,y}
    \end{equation}
    The flying foot acceleration in x and y is a linear function of p1 (the landing placement of the flying foot) thanks to the polynomial foot trajectory
    \begin{equation}
        [{J}_{flyingFoot} \ddot{q}+\dot{J}_{flyingFoot} \dot{q}]_{x,y}=[a_{flyingFoot}p_0+b_{flyingFoot}]_{x,y}
    \end{equation}
    \section*{Full problem}
    \begin{equation*}
    \begin{aligned}
    & \underset{\bold{p},\bold{\ddot{q}}}{\text{minimize}}
    & & \norm{
        \begin{pmatrix} 
             \bold{A}_{MPC x}     & 0 	 & 0\\
             0         & \bold{A}_{MPC y} & 0 \\
             0         & 0 &  \bold{A}_{WholeBody}
            \end{pmatrix}
        \begin{pmatrix} 
            \bold{p}_{x} \\
            \bold{p}_{y} \\
            \bold{\ddot{q}
            }
        \end{pmatrix}
        - \begin{pmatrix} 
            \bold{b}_{MPC x}\\
            \bold{b}_{MPC y}\\
            \bold{b}_{WholeBody}
        \end{pmatrix}	
    }_2 \\
    & \text{subject to}
    & & {\bold{A}_{constrains}\begin{pmatrix} 
            \bold{p} \\
            \bold{\ddot{q}}
        \end{pmatrix}- \bold{b}_{constrains}} 	
    \end{aligned}
    \end{equation*}

\section*{Work to be done}
Replace (\ref{min_step_dist}) with inequality.

Add double support phases

reimplement the polynomial trajectory generator.
\end{document}
